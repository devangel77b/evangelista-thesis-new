% (This is included by thesis.tex; you do not latex it by itself.)
\rcsid{$Id$}
\rcsid{$Header$}
\rcskwsave{$Author$}
\rcskwsave{$Date$} 
\rcskwsave{$Revision$}


\begin{abstract}
% The text of the abstract goes here.  If you need to use a \section
% command you will need to use \section*, \subsection*, etc. so that
% you don't get any numbering.  You probably won't be using any of
% these commands in the abstract anyway.
%Look at aerial righting and directed aerial descent in Chukar partridges.  In parallel, look at aerodynamic implications of different theropod and bird morphologies.  Consider some the effect of body size on damage and the effect of size and shape on turbulent noise pickup from turbulent flows.  Possibly to wrap up, weave all of these together in a phylogenetic framework.

This thesis consists of four main studies (all in preparation for publication):  a study of incipient flight behaviors in young birds over ontogeny (Chapter~\ref{ch:1}); a detailed study of maneuvering using physical models of a likely ancestral bird morphology (Chapter~\ref{ch:2}); a comparative study of maneuvering ability in several stem-group birds, within a phylogenetic context (Chapter~\ref{ch:3}); and development of basic engineering theory to quantify the turbulence sensitivity of shapes to environmental turbulence of given scales and spectral content (Chapter~\ref{ch:4}).  These are enabled by interdisciplinary application of engineering theory and techniques to the biomechanics of flight and aerial maneuvering.  The studies have identified: 1) shifts in function from asymmetric to symmetric movements in young birds, contrary to predictions from alternative hypotheses and occurring before wing-assisted incline running; 2) shifts in function, tied to angle of attack, of asymmetric appendage postures in creating yawing and rolling moments; and 3) migration of control effectiveness as tails are shortened and other features change, during the early evolution of birds. The work plugs some considerable gaps in current prevailing theories (e.g.\ \citealp{Dial:2003, Tobalske:2011}) and provides a test of hypotheses of flight evolution not based in outdated ``trees-down'' or ``ground-up'' paradigms from the past, but rather based on the universal need of airborne animals to maneuver.   

This thesis seeks to understand early flight evolution from a maneuvering perspective; every animal in the air must maneuver, and by understanding ``powered'' flight as simply a point along a spectrum of \emph{maneuvering} flight \citep{Dudley:2011}, unifying breakthroughs are made. This constitutes a re-examination of a major transition in vertebrate evolution in a way not considered before. The multifaceted approach taken, with ontogenetic series, aerodynamic studies, and phylogenetic approaches, is robust against the shortcomings of any one approach individually:  confounding ontogeny with evolution (as may be the case in others' studies of alternative hypotheses, e.g.\ \citep{Dial:2003}); inferring implausible functions from paleontological material in the absence of proper benchmarking against live animals; or misdiagnosis of how forms work in the absence of functional studies.  
\end{abstract}
