\rcsid{$Id$}
\rcsid{$Header$}
\rcskwsave{$Author$}
\rcskwsave{$Date$} 
\rcskwsave{$Revision$}

\chapter{Aerodynamic characteristics of a feathered dinosaur shape measured using physical models: effects of form on static stability and control effectiveness}
\label{ch:2}
\index{Microraptor|(}

%\begin{abstract}
We report the effects of posture and morphology on the static aerodynamic stability and control effectiveness of physical models based loosely on a feathered dinosaur, \Microraptorgui, from the Cretaceous of China.  While some leg postures render \Mgui\ unstable, and thus quick to maneuver, others are stable, slower to maneuver but resistant to perturbation by wind gusts.  Depending on body posture, asymmetric leg positions can cause roll but have surprisingly little effect on yaw, while raising and lowering the tail or the hind limbs can alter pitch.  These results may help bound speculation and inform debate regarding \Mgui\ specifically, which has attracted much attention due to its leg and tail feathers.  Furthermore, while \Mgui\ lived after \Archaeopteryx\ and likely represents a side experiment with feathered morphologies, the general patterns of stability and control effectiveness as leg and tail morphologies are changed may help understand the evolution of flight control aerodynamics in vertebrates.  As further fossils with different morphologies or postures are found, these results could be applied in a phylogenetic context to understand potential biomechanical constraints on extinct flyers or gliders arising from the need to maneuver. 
%\end{abstract}

\section{Introduction}
The evolution of flight in vertebrates, and particularly in birds, is the subject of lively debate and considerable speculation.  Furthermore, flight ability of extinct vertebrates is often inferred from very simple parameters (such as lift and drag coefficients and glide angles \cite{Emerson:1990, Emerson:1990b}); these alone may not be sufficient as anything flying in a real environment will experience perturbations and need to maneuver around obstacles \cite{Dudley:2012}.  

Discoveries during the last decade of a diversity of feathered dinosaurs and early birds from the Cretaceous of Liaoning, China have led to considerable speculation about the roles that the feathers played on these extinct animals. Fossil forms are important in biomechanical studies because they may indicate ``missing links'', transitional forms within a lineage, between ancestral and derived, or they record ``experiments'' in form in side-branches; both are informative for questions of biomechanics.  Although we cannot observe the behavior of extinct animals, we can measure the aerodynamic forces on dynamically-scaled physical models in a wind tunnel to quantify the broader effects on performance of different postures and morphologies.  Since physical laws apply the same to all taxa, regardless of history, knowing about the physical implications of shape can suggest ``priors'' that would apply to anyone in the air. 
  
The Jiufotang Formation has been interpreted as a forest based on pollen data and plant fragments \cite{someone}.  The inference that \Mgui\ was arboreal solely based on pollen is not terribly strong, given that not everything that lives in a forest lives in the trees and that processes after death (taphanomy) that occur during fossilization also tend to wash everything together.  However, quite many things in forests make use of the trees even if they don't appear particularly arboreal. In addition, the vertebrate diversity includes stuff as well as numerous feathered theropod dinosaurs and early birds \cite{someone}, many with small size and similar feathered forms, suggesting that at least some might have been in the trees and performing aerial behaviors even if \Mgui\ was not. Somehow this needs to be dealt with. See Xu letter and Zhou letter in Nature. 
  
We used such models, loosely based on \Microraptorgui\ (Figure~\ref{fig:mgui}), a cat-sized dromaeosaur with flight feathers on its forelimbs, hindlimbs, and tail, enabling us to investigate effects of diverse aerodynamic surfaces in the aft/posterior of a body.  By measuring not just lift and drag, but also side forces and moments in pitch, roll, and yaw, we can assess static aerodynamic stability (tendency to experience righting torques when perturbed) and control effectiveness (moments generated by motions of control surfaces), both of which affect the ability to maneuver while gliding or parachuting through a complex forest habitat. 
\begin{figure}
%\includegraphics[width=6in]{figures/fossil/Microraptor-IVPP-V13352-5cm.jpg}
\caption{%
{ \Microraptorgui\ \cite{Xu:2003}, a dromaeosaur from the Cretaceous Jiufotang Formation of Liaoning, China.}  Holotype specimen IVPP V13352, scale bar \SI{5}{\centi\meter}.  Notable features include semilunate carpal bones, a boomerang-shaped furcula, a shield-shaped sternum without a keel, uncinate processes on the ribs, unfused digits, an intermediate angle of the scapulocoracoid, and a long tail of roughly snout-vent length.  In addition, there are impressions of feathers on the forelimbs, hindlimbs, and tail.%
}
\label{fig:mgui} 
\end{figure}

We compared the lift, drag, and side forces, and the pitch, roll, and yaw moments on models with vs. without leg feathers, and we tested the models in different symmetric and asymmetric postures that have been proposed by various researchers.  In some cases leg feathers had no effect, and in others they did (e.g. leg feathers reduced drag for some postures at some angles of attack). Therefore, whether or not leg feathers affected gliding, parachuting, or maneuvering performance depended on the posture and orientation of the dinosaur.  These results will contribute to our understanding of the role of aerodynamic surfaces aft of a gliding animal's center of mass.

This is the first of two paper dealing with the \Mgui\ aerodynamics.  In this paper we do some stuff.  In the second paper, we will...

\subsection{Previous work with models}

Further lit review add in later.  Gatesy and Dial 1996 looked at Archaeopteryx tails, Longrich did some maneuvering calcs based on McCay; Chatterjee and Templin did purely computer model of phugoid dynamics; Nova guys, Alexander (2010) did full scale flying models.  Our methods are most similar to Xu Jenkins Breuer Nova and Alexander 2010.  

\subsection{Additional engineering background}

Multiple surfaces in tandem might be expected to have large impacts for maneuvering.  In engineering practice, submarine snap rolls can be caused by interactions between the sail and rudder (citation).  In biology, interactions between median or paired fins can enhance maneuvering in fish; the four-(or more) flipper planform is widely seen in aquatic creatures but seems largely absent from fliers (exceptions frogs, four-winged flying fish, any others?). Why?  

Changes in flight performance from being a crappy flier to a good flier might be expected to have changes in glide angle or AOA associated with them.  High AOA aerodynamics can be vastly different from low AOA, with shifts in stability expected.  Shifts in stability at high AOA is responsible for crashes (citations). 

In engineering practice there is also the concept of control plane reversal, in which a control surface acts the opposite of what it ``normally'' does; for example, at low speed ships rudders or planes acting opposite to their normal direction has caused collisions of cruise ships (citations) others (citations), 





% You may title this section "Methods" or "Models". 
% "Models" is not a valid title for PLoS ONE authors. However, PLoS ONE
% authors may use "Analysis" 
\section{Materials and Methods}

\subsection{Models and postures}
Scale models of \Mgui\ (%mass \SIrange{1}{1.4}{\kilo\gram}, 
snout-vent length \SI{8}{\centi\meter}) were constructed from published reconstructions and photographs \citep{Xu:2003, Chatterjee:2007, Nova}.  The models are shown in Figure~\ref{fig:model}A.  Model construction was guided by dissection of Starlings (\emph{Sturnus vulgaris}), reference to preserved specimens of birds, bird wings, and lizards, teaching casts of \Archaeopteryx, and illustrated textbooks on vertebrate functional morphology and vertebrate paleontology \citep{Liem:2000, Benton:1997}.  Photographs of the \Mgui\ holotype IVPP V13352 were printed on a laser printer (Xerox, Norwalk, CT) at full scale and at model scale to further guide model construction.  %However, no member of our research team has ever had the opportunity to personally examine an \Mgui\ fossil or cast.  
\begin{figure}
\caption{%
{ Physical models of \Mgui, wingspan \SI{20}{\centi\meter}, snout-vent-length \SI{8}{\centi\meter}.}  Reconstruction postures (a-d) used for constructing physical models:  a, sprawled, after \citep{Xu:2003};  b, tent, after \citep{Nova}; c, legs-down, after \citep{Nova};  d, biplane, after \citep{Chatterjee:2007}. Additional manipulations (e-h): e, sprawled without leg or tail feathers; f, tent without leg or tail feathers; g, example asymmetric leg posture with \ang{90} leg mismatch \emph{arabesque}; h, example asymmetric leg posture with \ang{45} dihedral on one leg \emph{d{\'e}gag{\'e}}.%
}
\label{fig:model}
\end{figure}

Models were built on an aluminum plate with polymer clay (Polyform Products Co., Elk Grove, IL) to fill out the body using methods described in \citep{Koehl:2012}.  Removable tails and heads, to allow repositioning, were constructed using polymer clay over steel rods.  The forelimbs were constructed by bending 26-gauge steel wire scaled to the lengths of the humerus, radius and ulna, and digits as seen in published photographs of the holotype.  Similarly, hindlimbs were constructed with wire scaled to the lengths of the femur, tibiotarsus, tarsometatarsus, and digits.  For the appendages and tail, feathered surfaces were modeled using paper and surgical tape (3M, St.\ Paul, MN) stiffened by addition of monofilament line at the locations of the individual feather rachises.  This method of creating wing surfaces was compared to wings with craft feathers individually sewn onto them and seen to provide equivalent results\citep{Koehl:2012}.  In addition, models of Anna's Hummingbirds (\emph{Calypte anna}) constructed using the same techniques have been shown to faithfully reproduce the aerodynamic properties of diving hummingbirds (Evangelista, in preparation).  

Model reconstruction postures (Figure~\ref{fig:model}B-E) were chosen based on those previously published \citep{Xu:2003, Chatterjee:2007, Nova}.  Some of these postures are anatomically dubious; in particular the sprawled posture drawn in \citep{Xu:2003} has been criticized because of interference between the trochanter on the femur and the surrounding structures of the ilium and ischium \citep{someone}, while a feasible mechanism for maintaining feathers in the biplane / muffed feet posture of \citep{Chatterjee:2007} under load has never been proposed.  We also tested models in postures more strongly inferred for theropods, including a legs-down posture with no more than \ang{45} leg abduction \citep{Nova}, and a tent posture in which the legs are extended caudad with the feathered surface extending over the proximal part of the tail.  %Xu never intended the sprawled posture as an actual reconstruction per se. \citep{Xu:2005a} check this, and in the absence of fossil material illustrating otherwise there is no reason to assume extraordinary hip anatomy not seen in any other theropod.  

We recognize that some of the reconstruction postures are less feasible than others.  The approach taken here is to test all previously proposed reconstructions, in order to examine the aerodynamic implications of these shapes from a purely physical standpoint.  With the uncertainties inherent in applying a physical modeling approach to an extinct animal with only a single published skeleton, statements about aerodynamic performance in \Mgui\ should always be taken with a grain of salt.  %There are other reasonable things said in numerous letters that ought to be mentioned here. 

%\subsection*{Conditions for dynamic similarity}
%To achieve dynamic similarity in these models, it would be nice to match the Reynolds number ($\mbox{Re} = uL/\nu$), the nondimensional ratio of viscous to inertial forces.  Based on pilot studies we estimated $\mbox{Re}$ for the full scale \Mgui\ to be approximately \num{200000}.   Limitations on the wind tunnel size and speed required the Reynolds number of the model to be \num{32000}.  How bad is this? Check that correct scales are being used and that the old uraps did the calculation right. 
%\input{tables/similarity.tex}

\subsection{Force measurements}
As described in \citep{Koehl:2012}, models were mounted on a six-axis force transducer (Nano17, ATI Industrial Automation, Apex, NC), which was in turn mounted on a 1/4-20 threaded rod damped with rubber tubing, and attached to a tripod head used to adjust angle-of-attack.  The force sensor and sting exited the model on the right side of the body mid-torso at approximately the center of mass.

Wind tunnel tests were conducted in an open jet wind tunnel with an \SI{15 x 15 x 18}{inch} (\SI{38.1 x 38.1 x 45.7}{\centi\meter}) working section used previously for studies of gliding frogs \citep{McCay:2001, McCay:2001a}.  Tunnel speed was controlled using a variable autotransformer (PowerStat, Superior Electric Company, Bridgeport, CT) and monitored using a hot wire anemometer (Series 2440, Kurz Instrument Co., Monterey, CA).  

Force transducer readings were recorded at \SI{1000}{\hertz} sampling frequency using a National Instruments 6251 data acquisition card (National Instruments, Austin, TX).  Raw measurements were rotated from a frame fixed to the model to one aligned with the wind tunnel and flow using the angle-of-attack.  Transformed measurements were then averaged over a one-minute recording.  For each measurement, wind tunnel speed $v$ was recorded and used to compute Reynolds number ($\mbox{Re} = vL/\nu$, $\nu = \SI{15.e-6}{\meter\squared\per\second}$).  The sign convention for forces and moments is shown in Figure~\ref{fig:signconvention}

\begin{figure}
%\includegraphics[width=4in]{figures/sign-conventions/Pitch-signs.pdf}
%\includegraphics[width=2in]{figures/sign-conventions/yaw-signs.pdf}
\caption{%
{Sign conventions, rotation angles, and definitions for model testing, after \citep{Emerson:1990b,McCay:2001,McCay:2001a,McCormick:1995}.}%
}
%\caption{}
\label{fig:signconvention}
\end{figure}  

Aerodynamics forces and moments were  normalized to obtain nondimensional coefficients according to the following (rewrite this to use notation from \citep{McCormick:1995}): 
\begin{equation}
\mbox{lift} = C_L 0.5 \rho u^2 A_p
%C_L = \frac{\mbox{lift}}{1/2 \rho u^2 A_p}
\end{equation}
\begin{equation}
\mbox{drag} = C_D 0.5 \rho u^2 A_p
%C_D = \frac{\mbox{drag}}{1/2 \rho u^2 A_p}
\end{equation}
\begin{equation}
\mbox{side force} = C_S 0.5 \rho u^2 A_p
%C_S = \frac{\mbox{side force}}{1/2 \rho u^2 A_p}
\end{equation}
\begin{equation}
\mbox{pitching moment} = C_m 0.5 \rho u^2 A_p \lambda_{SVL}
%C_m = \frac{\mbox{pitching moment}}{1/2 \rho u^2 A_p \lambda_{SVL}}
\end{equation}
\begin{equation}
\mbox{rolling moment} = C_r 0.5 \rho u^2 A_p \lambda_{SVL}
%C_r = \frac{\mbox{rolling moment}}{1/2 \rho u^2 A_p \lambda_{SVL}}
\end{equation}
\begin{equation}
\mbox{yawing moment} = C_y 0.5 \rho u^2 A_p \lambda_{SVL}
%C_y = \frac{\mbox{yawing moment}}{1/2 \rho u^2 A_p \lambda_{SVL}}
\end{equation}
where $\rho=\SI{1.204}{\kilo\gram\per\meter\cubed}$ is the air density, $A_p$ is the model planform area, and $\lambda_{SVL}$ is the snout-vent length of the model.  To allow comparisons among models, a single, consistent baseline configuration is needed.  Accordingly, nondimensional coefficients are referenced to the planform area of the four-winged, sprawled position originally proposed in \citep{Xu:2003} unless specially noted.  The questions of interest for this study are tied to the absolute value of forces and moments produced and differences that occur from the same animal in different postures; our choice of normalization preserves these distinctions in most cases.  

%MK:  I do remember deciding that we should use the full plan area of the body in the sprawled posture for all the postures with four wings.  I'm glad you let me know that you used the four-wing plan area to calculate the coefficients for the two-winged models.

%DE: 
%\emph{If we hit a case where we want to see area differences, we can show raw forces and coefficients using are but that is probably irrelevant for this study.  It comes up in Draco studies of no - half - full - double wing but is not important to the questions here.}

\subsection*{Static stability coefficients}
To assess static stability, we calculated nondimensional static stability coefficients from fixed-wing aircraft stability and control theory \citep{McCormick:1995} and previously used in studies of gliding frogs \citep{McCay:2001, McCay:2001a}.

The pitching stability coefficient $C_{m,\alpha}$ is defined as \citep{McCay:2001}
\begin{equation}
\partial C_m = C_{m,\alpha} \partial\alpha
%C_{m,\alpha} = \frac{\partial C_m}{\partial\alpha}
\end{equation} 
where $\alpha$ is the angle-of-attack and $C_m$ is the pitching moment coefficient as defined above. It is the local slope of the pitching moment curve, and is thus an indication of the sense (restoring or non-restoring) and magnitude of the torque generated in response to a perturbation in angle-of-attack.  If $C_{m,\alpha}<0$, the response torque will be opposite the direction of perturbation; this is the condition for static stability. 

Similarly, for roll:
\begin{equation}
\partial C_r = C_{r,\phi} \partial\phi
%C_{r,\phi} = \frac{\partial C_r}{\partial\phi}
\end{equation}
where $\phi$ is the roll angle and $C_{r,\phi}<0$ is the condition for static stability in roll. By symmetry, models at zero angle-of-attack have neutral rolling stability, and we did not calculate roll stability for most cases.

For yaw, 
\begin{equation}
\partial C_y = C_{y,\psi} \partial\psi
%C_{y,\psi} = \frac{\partial C_y}{\partial\psi}
\end{equation}
where $\psi$ is the yaw angle and $C_{y,\psi}<0$ is the condition for static stability in yaw (also known as directional stability).  

Pitching stability coefficients were obtained from angle-of-attack ($\alpha$) runs taken from \ang{-15} to \ang{90} at \ang{5} increments.  Yawing stability coefficients were obtained from yaw angle ($\psi$) runs from \ang{-30} to \ang{30} at \ang{10} increments. For each series, central differences were used to estimate the slopes at each point for each replicate run.   

\subsection{Control effectiveness}
We also calculated nondimensional control effectiveness coefficients using methods from aerodynamic engineering \citep{Etkin:1996} used in previous studies of gliding frogs \citep{McCay:2001a}. In general, control effectiveness for a control surface whose angular orientation relative to the flow can be changed is the partial derivative of the moment generated with respect to the angle. High control effectiveness means a large amount of moment generated for a small movement of the surface.  

We calculated the pitching control effectiveness for the tail, forewings, and legs as follows:
\begin{equation}
\partial C_m = C_{m,\delta} \partial\delta
%C_{m,\delta} = \frac{\partial C_m}{\partial\delta}
\end{equation}
where $\delta$ is the angle of the control surface in question.  Similarly, we calculated yawing control effectiveness for these surfaces as follows:
\begin{equation}
\partial C_y = C_{y,\delta} \partial\delta
%C_{y,\delta} = \frac{\partial C_y}{\partial\delta}
\end{equation}
as well as rolling control effectiveness for asymmetric movements of the wings and legs:
\begin{equation}
\partial C_r = C_{r,\delta} \partial\delta
%C_{r,\delta} = \frac{\partial C_r}{\partial\delta}
\end{equation}

\subsection{Other flight performance metrics}
To allow comparison with previous studies, two additional measures of maneuvering performance we computed: the banked turn maneuvering index and crabbed turn maneuvering index \citep{Emerson:1990b, McCay:2001, McCay:2001a}.  The banked turn maneuvering index assumes turns accomplished by banking is computed in two ways, both of which assume that some component of lift generated is used to provide the force necessary for turning:
\begin{equation}
MI_{\mbox{banked,1}} = \frac{C_{L,\mbox{max}}}{mg/A_P}
\end{equation}
after \citep{Emerson:1990b} (note this is not a nondimensional index), and 
\begin{equation}
MI_{\mbox{banked,2}} = \frac{L \cos{\phi}}{mg}
\end{equation}
where $\phi=\ang{60}$ is arbitrarily chosen with no reasonable basis for picking it, after \citep{McCay:2001, McCay:2001a}.  Similarly, for crabbed turns, a nondimensional index is the horizontal component of side force normalized by body weight \citep{McCay:2001, McCay:2001a}:
\begin{equation}
MI_{\mbox{crabbed}} = \frac{F_{\mbox{side}} \sin{\psi}}{mg}
\end{equation}
again with $\psi=\ang{60}$ arbitrarily chosen based on frogs \citep{McCay:2001,McCay:2001a}. 

Actually the indices here from \citep{Emerson:1990b, McCay:2001, McCay:2001a} are stupid; they are just recast versions of other numbers that are more informative without being mucked around with. 

We also calculated several flight performance metrics not immediately tied to maneuvering \citep{McCay:2001, McCay:2001a, Emerson:1990, Emerson:1990b}. As a measure of horizontal glide performance, we computed $(C_L/C_D)_{\mbox{max}}$ for each posture \citep{Emerson:1990b}. Minimum glide speed, a measure of the ease of which gliding can be initated, was also computed as $U_{\mbox{min}} = [2mg /(A_P \rho C_L)]^{1/2}$ \citep{Emerson:1990b}. As a measure of parachuting ability, we also compared $D_{90}$, the full scale drag for parachuting \citep{Emerson:1990b}, as well as a nondimensionalized parachuting index $D_{90}/mg$.  

%\subsection*{Agility and maneuverability}
%McCay quantified agility as the time (latency) to achieve an arbitrary \ang{60} change in yaw and used live animal kinematics or benchmarked simulations to estimate it \citep{McCay:2001a}.  Similarly, McCay defined maneuverability (other than the maneuvering indices above) as turning radii from live animal kinematics or benchmarked simulations \citep{McCay:2001a}.  Without a live animal or a simulation with sufficient benchmarking to have some confidence in its predictions, we did not compute either of these quantities for \Mgui.   

%\subsection*{Reynolds number sweep}
%A sweep in speed was conducted to check how much coefficients varied with speed.  

%\subsection*{Estimation of mass and gravitationally induced moments}
%The mass of a live \Mgui\ was estimated by scaling using data from some sources.  Fill in more later.  To estimate gravitationally induced moments and the location of the center of mass, mass of some body parts were scaled from birds and dimensions and positions estimated from fossil and reconstruction position.  Fill in more later. 






% Results and Discussion can be combined.
\section{Results}
During the fall of 2010, we collected a dataset of 12,810 measurements for 180 combinations of postures and positions.  The raw data require approximately 5.3 GB of storage.  The work was accomplished during approximately 350 hours of wind tunnel time by a team of ten undergraduates led by one graduate student.

For the plots given here, color represents the base posture: red for sprawled, blue for tent, green for biplane, and purple for down.  All sign conventions are as in \citep{McCay:2001} and as shown in Fig~\ref{fig:signconvention}.  Symbols, where used, represent variations in position from the base posture, such as movement of legs, wings, or tail. All units are SI unless otherwise noted. 

%\subsection*{Mass, center, and gravitational moment estimates?}
%\subsection*{Method and model benchmarking?}
%Raw force data, papers versus feather, tunnel speed calibration, Reynolds number sweep mostly covered in stupid ICB paper.

\subsection{Baseline longitudinal plane aerodynamic data and effects of posture and the presence/absence of leg and tail feathers}
Fig~\ref{fig:coeffsvsaoa} gives the nondimensional coefficients of lift, drag, and pitching moment for \Mgui with full feathers.  
\begin{figure}
a %\includegraphics{figures/Gliding/pitch-stability/basicClvsaoa.pdf}
b %\includegraphics{figures/Gliding/pitch-stability/basicCdvsaoa.pdf}
c %\includegraphics{figures/Gliding/pitch-stability/basicClvsCd.pdf}
d %\includegraphics{figures/Gliding/pitch-stability/basicCmvsaoa.pdf}
\caption{{ Nondimensional coefficients for all models.}  Red is sprawled, blue is tent, green is biplane, purple is down. $\alpha$ from \ang{-15} to \ang{90} in \ang{5} increments, with 5 or more replicates per treatment. a: Lift coefficent. b: Drag coefficient. c: Lift drag polars.  d: Pitching moment coefficient.}
\label{fig:coeffsvsaoa}
\end{figure}
Scaling with the coefficients, the full scale forces for \Mgui\ at \SI{12}{\meter\per\second} are plotted in Fig~\ref{fig:fsall}.
\begin{figure}
a %\includegraphics{figures/Gliding/pitch-stability/basicfsLvsaoa.pdf}
b %\includegraphics{figures/Gliding/pitch-stability/basicfsDvsaoa.pdf}
c %\includegraphics{figures/Gliding/pitch-stability/basicfsLvsD.pdf}
d %\includegraphics{figures/Gliding/pitch-stability/basicfsMvsaoa.pdf}
\caption{{ Full scale forces and moments for \Mgui\ at \SI{12}{\meter\per\second}}.  Red is sprawled, blue is tent, green is biplane, purple is down. $\alpha$ from \ang{-15} to \ang{90} in \ang{5} increments, with 5 or more replicates per treatment. a: Full scale lift at \SI{12}{\meter\per\second}, all models. This figure must be annotated to show the band of \Mgui\ body weight. b:  Full scale drag at \SI{12}{\meter\per\second}, all models. This figure must be annotated to show the band of \Mgui\ body weight. c: Lift-drag polars. d: Full scale pitching moment at \SI{12}{\meter\per\second} versus angle of attack, all models.}
\label{fig:fsall}
\end{figure}
For comparison with previous work \citep{Emerson:1990b}, various other gliding performance metrics are compared in figures~\ref{fig:EKcomparisons1} and \ref{fig:EKcomparisons2}.  
\begin{figure}
a %\includegraphics{figures/Gliding/pitch-stability/liftdragratio.pdf}
b %\includegraphics{figures/Gliding/pitch-stability/glideangle.pdf}
c %\includegraphics{figures/Gliding/pitch-stability/Umin.pdf}
%d \includegraphics{figures/Gliding/pitch-stability/Upropelled.pdf}
d %\includegraphics{figures/Gliding/pitch-stability/Uterm.pdf}
e %\includegraphics{figures/Gliding/pitch-stability/pitchstab.pdf}   
\caption{ Red is sprawled, blue is tent, green is biplane, purple is down. $\alpha$ from \ang{-15} to \ang{90} in \ang{5} increments, with 5 or more replicates per treatment. a: Lift to drag ratio. b: Glide angle. c: Minimum glide speed. d: Terminal velocity (assuming stability). e: Pitching stability coefficient (note pitching moment must also be zero for stable equilibrium).}
\label{fig:EKcomparisons1}
\end{figure}
\begin{figure}
a %\includegraphics{figures/Gliding/emerson-koehl-1/ClCdvsposture.pdf}
b %\includegraphics{figures/Gliding/emerson-koehl-1/Uminvsposture.pdf}
c %\includegraphics{figures/Gliding/emerson-koehl-1/D90vsposture.pdf}
\caption{%
{ Comparison of simple glide metrics after \citep{Emerson:1990b} suggests the metrics are not informative.}
Red is sprawled, blue is tent, green is biplane, purple is down.%
a:  Maximum lift to drag ratio, by posture, without regard to stability.  \citep{Emerson:1990b}'s minimum ratio is never achieved because the models are not stable at the point where $L/D$ is maximum.  There is no difference in maximum lift to drag ratio among postures (Kruskal-Wallis, $P=0.1740$). b: Minimum glide initiation speed, by posture, without regard to stability. The minimum speed is never achieved because the models are not stable at the point where $U_{min}$ is lowest.  There is no difference in $U_{min}$ among postures (Kruskal-Wallis, $P=0.575$).  c:  Parachuting drag, by posture, without regard to stability. This drag is never achieved because the models are not stable at a \ang{90} angle-of-attack.  There are significant differences in $D_{90}$ among postures (Kruskal-Wallis, $P=\num{9.2e-5}$); sprawled position has higher parachuting drag.}
\label{fig:EKcomparisons2}
\end{figure}
A Reynolds number sweep (Fig~\ref{fig:Reynoldsnothing}, Table~\ref{tbl:similarity}) was also conducted to check for scale effects. 
\begin{figure}
%\includegraphics{figures/Methods/reynolds-sweeps/liftvsRe.pdf}\\
%\includegraphics{figures/Methods/reynolds-sweeps/dragvsRe.pdf}\\
%\includegraphics{figures/Methods/reynolds-sweeps/pitchvsRe.pdf}
\caption{Reynolds number sweeps for lift, drag, and pitch coefficients. There are not large changes in aerodynamic coefficients over the ranges shown here.  This is similar to what is seen in \emph{Draco} lizard and Anna's Hummingbird (\emph{Calypte anna}) models. The coefficients are roughly constant in the range of \emph{\dag Archaeopteryx} and are constant enough for these results to be applicable to \emph{\dag Microraptor}.}
\label{fig:Reynoldsnothing}
\end{figure}
%\input{tables/similarity.tex}

%\subsection*{Effect of leg and tail feathers}
The effects on longitudinal plane coefficients of the presence or absence of leg and tail feathers is shown in Fig~\ref{fig:feathers1} and \ref{fig:feathersEK}, as a ``what if'' examination of morphological changes that may put the results in context.  
\begin{figure}
a %\includegraphics{figures/Gliding/adrian-eric-pitch-stability2/aeClvsaoa.pdf}
b %\includegraphics{figures/Gliding/adrian-eric-pitch-stability2/aeCdvsaoa.pdf}
c %\includegraphics{figures/Gliding/adrian-eric-pitch-stability2/aeClvsCd.pdf}
d %\includegraphics{figures/Gliding/adrian-eric-pitch-stability2/aeLDvsaoa.pdf}
e %\includegraphics{figures/Gliding/adrian-eric-pitch-stability2/aeCmvsaoa.pdf}
f %\includegraphics{figures/Gliding/adrian-eric-pitch-stability2/aepitchstab.pdf}
\caption{{ Presence or absence of leg and tail feathers can drastically alter longitudinal plane aerodynamics.}.  Sprawled and tent postures with and without feathers. All coefficients shown versus angle-of-attack. a: Lift coefficient.  Stall occurs at higher angle-of-attack when leg feathers are present. b: Drag coefficient.  Leg feathers increase drag at high angle-of-attack, improving parachuting performance. c: Lift coefficient versus drag coefficient. d: Lift to drag ratio. Lift to drag ratio is improved slightly without the additional drag and less-efficient lift generation of hind wings. e: Pitching moment coefficient. Without leg feathers, stability is not achieved in either posture. f: Pitching stability coefficient.}
\label{fig:feathers1}
\end{figure}
\begin{figure}
%\includegraphics{figures/Gliding/emerson-koehl-2/ClCdvsposturefeathers.pdf}
%\includegraphics{figures/Gliding/emerson-koehl-2/Uminvsposturefeathers.pdf}
%\includegraphics{figures/Gliding/emerson-koehl-2/D90vsposturefeathers.pdf}
\caption{{ Presence or absence of leg and tail feathers has effects on \citep{Emerson:1990b} metrics}.  a:  Maximum lift to drag ratio, by sprawled and tent postures with and without feathers.  The maximum lift to drag ratio for tent without leg or tail feathers is significantly higher than for other postures (ANOVA, $P < 0.003$), however, this improvement is never achieved because the tent posture is never stable without leg feathers. b:  Minimum glide speed, by sprawled and tent postures with and without feathers.  There are no differences in minimum glide speed between postures (ANOVA, $P > 0.08$). c:  Parachuting drag, by sprawled and tent postures with and without feathers.  There are significant differences in parachuting drag between postures (ANOVA, $P < 0.04$), however, the straight-down parachuting position is not stable in any posture. }
\label{fig:feathersEK}
\end{figure}


\subsection*{A closer look at stability, control effectiveness, and maneuvering}
Perhaps from here and below will go in Part 2???
\subsection{Yaw stability}
\begin{figure}
%\includegraphics{figures/Gliding/yaw-stability/yaw-0/Cyawvshdg.pdf}\\
%\includegraphics{figures/Gliding/yaw-stability/yaw-0/Csidevshdg.pdf}
\caption{At \ang{0} angle-of-attack, there are clear differences in yaw stability between postures.  In particular, with legs down, the legs strongly act as weathervanes to stabilize the body in yaw. }
\label{fig:yaw1}
\end{figure}
\begin{figure}
\ang{0} %\includegraphics{figures/Gliding/yaw-stability/yaw-0/Cyawvshdg.pdf}\\
\ang{60} %\includegraphics{figures/Gliding/yaw-stability/yaw-60/Cyawvshdg.pdf} \\
\ang{90} %\includegraphics{figures/Gliding/yaw-stability/yaw-90/Cyawvshdg.pdf} \\
\caption{There are also clear differences in yaw stability at different angles-of-attack. At \ang{0}, some postures are more stable in yaw than others.  At \ang{60}, postures that were stable at \ang{0} may go unstable, such as tent posture.  At \ang{90}, all postures are marginally stable due to symmetry. }
\label{fig:yaw2}
\end{figure}
\begin{figure}
\ang{0} %\includegraphics{figures/Gliding/adrian-eric-yaw-stability/aeyaw-0/aeCyawvshdg.pdf}\\
\ang{60} %\includegraphics{figures/Gliding/adrian-eric-yaw-stability/aeyaw-60/aeCyawvshdg.pdf} \\
\ang{90} %\includegraphics{figures/Gliding/adrian-eric-yaw-stability/aeyaw-90/aeCyawvshdg.pdf} \\
\caption{The differences in yaw stability at different angles-of-attack also depend on the presence or absence of leg feathers. At \ang{0}, some feathered-leg postures are more stable in yaw than others.  At \ang{60}, postures that were stable at \ang{0} may go unstable, such as tent posture with leg feathers.  At \ang{90}, all postures are marginally stable due to symmetry. }
\label{fig:yaw3}
\end{figure}




\subsection{Control effectiveness of tail, symmetric wing and leg movements}
\begin{figure}
%\includegraphics{figures/Gliding/control-effectiveness-1/tail/biplane/Clvsaoa.pdf} \\
%\includegraphics{figures/Gliding/control-effectiveness-1/tail/biplane/Cmvsaoa.pdf} \\
\caption{Tail control effectiveness for biplane posture for tail angles of \ang{-15} (down triangle), \ang{0} (square), and \ang{+15} (up triangle).  At low angle-of-attack, tail up produces a nose up moment relative to zero tail angle, while tail down produces a nose down moment relative to zero tail angle.  The small effect on lift suggests the tail is primarily effective because of moments generated by its long length.}
\end{figure}
\begin{figure}
%\includegraphics{figures/Gliding/control-effectiveness-1/tail/down/Clvsaoa.pdf} \\
%\includegraphics{figures/Gliding/control-effectiveness-1/tail/down/Cmvsaoa.pdf} \\
\caption{Tail control effectiveness for down posture for tail angles of \ang{-15} (down triangle), \ang{0} (square), and \ang{+15} (up triangle).  At low angle-of-attack, tail up produces a nose up moment relative to zero tail angle, while tail down produces a nose down moment relative to zero tail angle.  At high angle of attack, the tail experiences reversal in which tail down produces nose up moments / tail up produces nose down moments.}
\end{figure}
\begin{figure}
%\includegraphics{figures/Gliding/control-effectiveness-1/tail/sprawled/Clvsaoa.pdf} 
%\includegraphics{figures/Gliding/control-effectiveness-1/tail/sprawled-no-feathers/Clvsaoa.pdf} \\
%\includegraphics{figures/Gliding/control-effectiveness-1/tail/sprawled/Cmvsaoa.pdf} 
%\includegraphics{figures/Gliding/control-effectiveness-1/tail/sprawled-no-feathers/Cmvsaoa.pdf} \\
\caption{Tail control effectiveness for sprawled posture for tail angles of \ang{-15} (down triangle), \ang{0} (square), and \ang{+15} (up triangle).  At low angle-of-attack, tail up produces a nose up moment relative to zero tail angle, while tail down produces a nose down moment relative to zero tail angle.  Reversal is not seen at high angle-of-attack.  Without leg feathers, the tail is ineffective at producing lift or pitching moment.}
\end{figure}
\begin{figure}
%\includegraphics{figures/Gliding/control-effectiveness-1/tail/tent/Clvsaoa.pdf} 
%\includegraphics{figures/Gliding/control-effectiveness-1/tail/tent-no-feathers/Clvsaoa.pdf} \\
%\includegraphics{figures/Gliding/control-effectiveness-1/tail/tent/Cmvsaoa.pdf} 
%\includegraphics{figures/Gliding/control-effectiveness-1/tail/tent-no-feathers/Cmvsaoa.pdf} \\
\caption{Tail control effectiveness for tent posture for tail angles of \ang{-30} (large down triangle),\ang{-15} (down triangle), \ang{0} (square), \ang{+15} (up triangle), and \ang{+30} (large up triangle).  At low angle-of-attack, tail up produces a nose up moment relative to zero tail angle, while tail down produces a nose down moment relative to zero tail angle.  Some reversal occurs at high angle-of-attack.  Without leg feathers, the tail is ineffective at producing lift or pitching moment.}
\end{figure}
\begin{figure}
%\includegraphics{figures/Gliding/control-effectiveness-1/legs/sprawled/Clvsaoa.pdf} \\
%\includegraphics{figures/Gliding/control-effectiveness-1/legs/sprawled/Cmvsaoa.pdf} 
\caption{Leg control effectiveness for sprawled posture for leg angles of \ang{-15} (down triangle), \ang{0} (square), and \ang{+15} (up triangle).  At low angle-of-attack, legs up produces a nose up moment relative to zero leg angle, while legs down produces a nose down moment relative to zero leg angle. Leg movement is ineffective at high angle of attack.}
\end{figure}
\begin{figure}
%\includegraphics{figures/Gliding/control-effectiveness-1/legs/tent/Clvsaoa.pdf} 
%\includegraphics{figures/Gliding/control-effectiveness-1/legs/tent-no-feathers/Clvsaoa.pdf} \\
%\includegraphics{figures/Gliding/control-effectiveness-1/legs/tent/Cmvsaoa.pdf} 
%\includegraphics{figures/Gliding/control-effectiveness-1/legs/tent-no-feathers/Cmvsaoa.pdf} \\
\caption{Leg control effectiveness for tent posture for leg angles of \ang{-30} (large down triangle),\ang{-15} (down triangle), \ang{0} (square), \ang{+15} (up triangle), and \ang{+30} (large up triangle).  At low angle-of-attack, leg up produces a nose up moment relative to zero leg angle, while leg down produces a nose down moment relative to zero leg angle. Without leg feathers, the legs still have some effect?  This plot may have bad data in it.}
\end{figure}
\begin{figure}
%\includegraphics{figures/Gliding/control-effectiveness-1/wings/tent-sweep/Clvsaoa.pdf} \\
%\includegraphics{figures/Gliding/control-effectiveness-1/wings/tent-sweep/Cmvsaoa.pdf} 
\caption{Symmetric wing sweep control effectiveness for tent posture for wing sweep angles of \ang{-45} (large down triangle), \ang{-22.5} (down triangle), \ang{0} (square), \ang{+22.5} (up triangle) and \ang{+45} (large up triangle).  Wing sweep is very effective at generating pitching moments.  Forward sweep generates nose up moments, while backwards sweep generates nose down moments.  This is like steering a wind surfing rig and is similar to what is seen in Anna's Hummingbird dive models (Evangelista, in preparation). This mode of control does not exhibit reversal.}
\end{figure}
\begin{figure}
%\includegraphics{figures/Gliding/control-effectiveness-1/wings/tent-pronation/Clvsaoa.pdf} \\
%\includegraphics{figures/Gliding/control-effectiveness-1/wings/tent-pronation/Cmvsaoa.pdf} 
\caption{Symmetric wing pronation/supination control effectiveness for tent posture for wing angles of \ang{-30} (large down triangle), \ang{-15} (down triangle), \ang{0} (square), \ang{+15} (up triangle) and \ang{+30} (large up triangle).  Wing pronation/supination (wing angle-of-attack) is effective at changing the lift generated but exhibits reversal at high angle-of-attack where stall occurs.}
\end{figure}

\eject



\subsection{Control effectiveness of asymmetric wing positions}
\begin{figure}
%%\includegraphics{figures/Gliding/control-effectiveness-2/sweep/Clvsaoa.pdf} \\
%%\includegraphics{figures/Gliding/control-effectiveness-2/sweep/Cdvsaoa.pdf} \\
%\includegraphics{figures/Gliding/control-effectiveness-2/sweep/Cmvsaoa.pdf} \\
%\includegraphics{figures/Gliding/control-effectiveness-2/sweep/Crollvsaoa.pdf} \\
%%\includegraphics{figures/Gliding/control-effectiveness-2/sweep/Cyawvsaoa.pdf} \\
\caption{Asymmetric wing sweep control effectiveness for tent posture for wing sweep angles of \ang{-45} (large down triangle), \ang{-22.5} (down triangle), \ang{0} (square), \ang{+22.5} (up triangle) and \ang{+45} (large up triangle). Forward sweep generates upward pitching moments, backward sweep generates downoard pitching moments.  Considerable roll moments are also generated at higher angles-of-attack.}
\end{figure}
\begin{figure}
%%\includegraphics{figures/Gliding/control-effectiveness-2/pronation/Clvsaoa.pdf} \\
%%\includegraphics{figures/Gliding/control-effectiveness-2/pronation/Cdvsaoa.pdf} \\
%%\includegraphics{figures/Gliding/control-effectiveness-2/pronation/Cmvsaoa.pdf} \\
%\includegraphics{figures/Gliding/control-effectiveness-2/pronation/Crollvsaoa.pdf} \\
%\includegraphics{figures/Gliding/control-effectiveness-2/pronation/Cyawvsaoa.pdf} \\
\caption{Asymmetric wing pronation control effectiveness for tent posture for wing pronation angles of \ang{-30} (large down triangle), \ang{-15} (down triangle), \ang{0} (square), \ang{+15} (up triangle) and \ang{+30} (large up triangle). At low angles-of-attack, asymmetric wing pronation generates large rolling moments.  At high angles-of-attack, there is a shift in function and asymmetric wing pronation tends to generate yawing moments instead of rolling moments.}
\end{figure}
\begin{figure}
%\includegraphics{figures/Gliding/control-effectiveness-2/tuck/Clvsaoa.pdf} \\
%%\includegraphics{figures/Gliding/control-effectiveness-2/tuck/Cdvsaoa.pdf} \\
%%\includegraphics{figures/Gliding/control-effectiveness-2/tuck/Cmvsaoa.pdf} \\
%\includegraphics{figures/Gliding/control-effectiveness-2/tuck/Crollvsaoa.pdf} \\
%\includegraphics{figures/Gliding/control-effectiveness-2/tuck/Cyawvsaoa.pdf} \\
\caption{Asymmetric wing tucking control effectiveness for tent posture; both wings out (solid square), no right wing (open square) and no wings (open diamond). Tucking one wing produces large roll moments but at the expense of one quarter of the lift.  Large yaw moments are not generated except at higher angles-of-attack where the leg and tail positions become more important.}
\end{figure}
\subsection{Control effectiveness of asymmetric leg positions in yaw}
Control effectiveness of asymmetric leg positions in yaw is plotted below.
\begin{figure}
\ang{0} %\includegraphics[width=2in]{figures/Gliding/control-effectiveness-3/leg-dihedral/0/Cyawvshdg.pdf}
%\includegraphics[width=2in]{figures/Gliding/control-effectiveness-3/leg-dihedral/0/Csidevshdg.pdf}
%\includegraphics[width=2in]{figures/Gliding/control-effectiveness-3/leg-dihedral/0/Crollvshdg.pdf} \\
\ang{30} %\includegraphics[width=2in]{figures/Gliding/control-effectiveness-3/leg-dihedral/30/Cyawvshdg.pdf}
%\includegraphics[width=2in]{figures/Gliding/control-effectiveness-3/leg-dihedral/30/Csidevshdg.pdf}
%\includegraphics[width=2in]{figures/Gliding/control-effectiveness-3/leg-dihedral/30/Crollvshdg.pdf} \\
\ang{60} %\includegraphics[width=2in]{figures/Gliding/control-effectiveness-3/leg-dihedral/60/Cyawvshdg.pdf}
%\includegraphics[width=2in]{figures/Gliding/control-effectiveness-3/leg-dihedral/60/Csidevshdg.pdf}
%\includegraphics[width=2in]{figures/Gliding/control-effectiveness-3/leg-dihedral/60/Crollvshdg.pdf} \\
\ang{90} %\includegraphics[width=2in]{figures/Gliding/control-effectiveness-3/leg-dihedral/90/Cyawvshdg.pdf}
%\includegraphics[width=2in]{figures/Gliding/control-effectiveness-3/leg-dihedral/90/Csidevshdg.pdf}
%\includegraphics[width=2in]{figures/Gliding/control-effectiveness-3/leg-dihedral/90/Crollvshdg.pdf} \\
\caption{Asymmetric leg dihedral (leg \emph{degage}) effect on yaw.  Baseline down position (solid square) versus one leg at \ang{45} dihedral (down arrow).  Placing one leg at a dihedral is destabilizing in yaw and produces side force and yawing moments.}
\end{figure}
\begin{figure}
\caption{Asymmetric leg dihedral (leg \emph{degage}) effect on yaw, without leg feathers.  Baseline down position (solid square) versus one leg at \ang{45} dihedral (down arrow).  Without leg feathers, nothing happens in this position.}
\end{figure}
\begin{figure}
\caption{Asymmetric one leg down (leg \emph{arabesque}) effect on yaw.  Baseline tent position (solid square) versus one leg at \ang{90} mismatch (down arrow).  Placing one leg down has little effect.}
\end{figure}
\begin{figure}
\caption{Asymmetric one leg down (leg \emph{arabesque}) effect on yaw without leg feathers.  Baseline tent position (solid square) versus one leg at \ang{90} mismatch (down arrow).  Placing one leg down had little effect; with no leg or tail feathers there is no effect.}
\end{figure}
\begin{figure}
\caption{Asymmetric tail movement effect on yaw, down posture.  Baseline down position (solid square), tail \ang{10} left (open square), tail \ang{20} left (open triangle), tail \ang{30} left (open diamond). The tail is effective at creating yawing moments but at low angles-of-attack it is shadowed by the body and larger movements are needed.}
\end{figure}
\begin{figure}
\caption{Asymmetric tail movement effect on yaw, tent posture.  Baseline tent position (solid square), tail \ang{10} left (open square), tail \ang{20} left (open triangle), tail \ang{30} left (open diamond). The tail is effective at creating yawing moments but at low angles-of-attack it is shadowed by the body and larger movements are needed.}
\end{figure}
\begin{figure}
\caption{Asymmetric tail movement effect on yaw, down posture.  Baseline down position (solid square), tail \ang{10} left (open square), tail \ang{20} left (open triangle), tail \ang{30} left (open diamond). The tail is effective at creating yawing moments but at low angles-of-attack it is shadowed by the body and larger movements are needed.}
\end{figure}
\begin{figure}
\caption{Asymmetric one wing down effect on yaw, tent posture.  Baseline tent position (solid square), left wing down (down triangle). Placing one wing down does not make large yawing moments.  Some roll and side force is produced at low angles-of-attack, at the expense of one quarter to one half of the lift.}
\end{figure}







\section{Discussion}

\subsection{Postures have similar lift and drag coefficients but exhibit very different stability}
All postures have roughly similar lift coefficients at low angles-of-attack (Fig.~\ref{fig:coeffsvsaoa}a); at high angles-of-attack, the main differences are due to the orientation and projected area of the legs.  

Examining the pitching moments reveals that only the biplane and tent postures have stable points (Fig.~\ref{fig:coeffsvsaoa}c).  For the tent position, the stable glide angle is \ang{35}, at roughly \SI{12}{\meter\per\second} and an angle-of-attack of \ang{27}.  For the biplane positions... The sprawled posture, which possesses roughly equal fore and aft area, is marginally stable in pitch (in effect, the longitudinal center of pressure is at the center of mass), while the down posture is never stable because the legs are not employed in lift generation (the longitudinal center of pressure is ahead of the center of mass).  

These stability results agree with \citep{Chatterjee:2007}, who argued from (something) that the biplane posture was stable. In contrast, Xu et al. \citep{Nova}, found the biplane to be unstable except at high angle-of-attack, however, without those results to review it is not possible to comment why. Xu et al. \citep{Nova} also found the tent posture to be stable, which agrees with our results. Alexander \citep{Alexander:2010} found that with nose-heavy ballasting, a sprawled/biplane posture could be made stable; we agree with this, with the caveat that such ballasting may not be biologically realistic as the densities of biological tissues do not vary as greatly as the density difference between lead and styrofoam.  

Our predicted equilibrium glide angle for the tent position seems reasonable given (comparison data here).  The animal would be fast enough to require some kind of landing maneuver to avoid injury; using an approach similar to Tedrake (citation), one could evaluate the perching or landing ability of this animal using our data.  Our glide angle and speed are higher than in Alexander \citep{Alexander:2010}, however, their weight estimate is half ours, and their models were constructed from model airplane parts that were already designed to fly.

Based on projected full scale forces (Fig.~\ref{fig:fsall}) and stability considerations, we estimate the \Mgui\ could glide in tent position.  Sprawled posture and down posture are unstable in pitch.  Biplane position does not appear to generate sufficient lift.  We did not mechanically evaluate if feathers cantilevered out the feet in the style of muffed feet on pigeons is able to carry significant loads; however this was a common point of failure in our models suggesting it would have been a limitation for that hypothetical posture.    

At first glance, there also appear to be differences in the maximum lift to drag ratio, minimum glide initiation speed, and parachuting drag for different postures (Figs.~\ref{fig:EKcomparisons1} and \ref{fig:EKcomparisons2}). It is important to note that these ``optima'' reflect a very narrow criteria of optimality and are not always achievable because of constraints, such as from stability or anatomy.  In particular, none of the most ``optimal'' configurations are stable.  Blind application of gross aerodynamic performance parameters (such as \citep{Emerson:1990b} may be misleading if it ignores such other constraints.  

%\subsection*{Coefficients are insensitive to Reynolds number}
The Reynolds number sweep (Fig~\ref{fig:Reynoldsnothing}, Table~\ref{tbl:similarity}) shows that the models under test here are in a regime where aerodynamic coefficients are relatively insensitive to Reynolds number, so that results are valid for the full-scale \Mgui, as well as for full-scale \Archaeopteryx.  This result was briefly discussed in \citep{Koehl:2012} but additional details are relevant here.  Unlike in gliding ants (Munk, in preparation) or in typical low Reynolds number structures such as crab antennas (Waldrop, in preparation) or copepod feeding appendages (Koehl citation), there are not dramatic shifts in function of the wings as Reynolds number is varied over a range of sizes and speeds (Fig.~\ref{fig:Reynoldsnothing}).  This is similar to what is observed in wind tunnel models of \Draco\ lizards (Evangelista, in preparation) and Anna's Hummingbirds (\emph{Calypte anna}) (Evangelista, in preparation) and similar to what is expected from typical high Reynolds number aerodynamics (citation). The result here suggests that these principles may also be applicable in evaluating maneuvering in juvenile birds during ontogeny.  

%\subsection*{Leg and tail feathers have important implications for aerodynamics and stability}
Leg feathers forming a hindwing will experience delayed onset of stall (Fig.~\ref{fig:feathers1}a, similar to a jib and a mainsail, or flaps on an airliner), increase drag at high angles-of-attack (Fig.~\ref{fig:feathers1}b), and drastically alter stability (Fig.~\ref{fig:feathers1}d.  None of the shapes tested were stable without leg feathers present (Fig.~\ref{fig:feathers1}d).  This suggests that leg/tail morphology in fossils may be informative as to the stable glide angles or positions an organism can adopt in the air. The leg feathers were initially downplayed in criticism as a taphonomic artifact; however subsequent finds of a wealth of specimens with feathers on the legs beg further work to evaluate their aerodynamic significance in a comparative framework. 

Leg feathers increased $D_{90}$ and decreased the lift to drag ratio, however, without leg feathers the models were not stable (Fig.~\ref{fig:feathersEK}). Higher L/D without leg feathers may be achieved by reduced drag from surfaces whose ability to produce lift is limited by their downstream location behind the forewings.  This may be a reason to shift from a feathered leg form to a larger forewing reduced leg form (as is seen in the evolution of birds).  For \Mgui\ in tent position with no leg feathers, it ought to glide slightly shallower, at the expense of having to go 1.4x faster (about \SI{17}{\meter\per\second}) and requiring some invisible hand of optimal but unachievable L/D ratios to stabilize it.  In reality, it would pitch upwards until stalling, and then crash and die.  This illustrates once again that assuming ``better glide performance'' is a single number such as L/D is an oversimplification; higher L/D means only lower steady glide angle when there's no guarantee an animal actually uses such trajectories (\emph{Draco}?); high L/D does not mean lower glide speed, and a high L/D may be unachievable because of constraints from stability or anatomy.





 



\subsection{Yaw stability depends on posture and leg feathers, and exhibits shifts based on angle-of-attack}
Perhaps here and below will go in Part 2?

Some postures (notably down) were observed to be more stable than others (Fig.~\ref{fig:yaw1}).  More importantly, postures which are stable at low angle-of-attack (such as tent) were unstable at intermediate angle-of-attack, and all postures were marginally stable at \ang{90} angle-of-attack (Fig.~\ref{fig:yaw2}.  Leg feathers were similarly seen to have different effects on stability with angle-of-attack (Fig.~\ref{fig:yaw3}).  The significance of this result is that during a shift from parachuting, through mid-AOA gliding, to low-AOA flight, different plan forms have drastically different stability characteristics in yaw.  The aerodynamic basis for this is not yet clear, though likely due to effects of vortex shedding or separation.  Further work is needed to examine this using flow visualization. 

\subsection{Control effectiveness varies with AOA and can exhibit reversal or shifts from one axis to another}
Control effectiveness was observed to vary with AOA (Figs. 42 onwards); furthermore there were cases in which its sign completely switched, i.e. when a control surface does the opposite of what it normally does (Fig. 43, down posture with the tail in pitch; Fig. 45, tent posture with the tail in pitch; Fig. 49 wing pronation in tent posture).  These happen in pitch at high angles of attack and in yaw at different angles of attack and postures.  Reversal during abnormal operating conditions in vehicles/ships etc can cause collisions and crashes; in a biological system it is a complete shift in function that happens in some interesting operating condition (like steep gliding versus shallow gliding). This deserves further study; the basis for reversal is unclear in these models and flow visualization is needed.   

As with our other measurements, removal of leg feathers tended to drastically reduce control effectiveness (for example, Fig. 44).  This might suggest that as birds evolved and moved away from long tails and feathered legs, the control effectiveness that those surfaces once possessed became reduced, or possibly was shifted to another surface (the wings).  This is bolstered by the observation that birds with partially amputated tails can still fly (citation).  In our data, wing sweep (in a manner similar to steering a windsurfing rig) was very effective at creating pitching moments.  

For asymmetric movements, similar trends were observed.  In particular, tent posture with one wing changing its pronation/supination was observed to produce large rolling moments at low AOA but large yawing moments at high AOA (Fig. 51).  This is another observation of a major shift in the function of a control surface with AOA.  

On the other hand, some movements (such as tucking one wing (Fig. 52) in or placing one wing down (Fig. 60)) did surprisingly little for yaw, roll or side force, at the tremendous expense of the loss of a large portion of lift.  There is no basis for such postures in the fossils and no basis for such postures in the flight of any extant creature, and no further work is needed on these. 










\subsection{Maneuvering must be considered when considering the evolution of flight in vertebrates}
Taken together, these results show that morphology can have large effects on the stability and control effectiveness and also place constraints on aerodynamic performance as to if reduced glide angles, lower glide speeds, or improved parachuting performance can be achieved. The changes in tail and leg morphology during the transition from theropods to birds (and convergent changes from early pterosaurs to later pterosaurs and early bats to later bats) beg for the metrics observed here to be studied in a comparative context, to examine how they change as the morphologies are changed and to examine what skeletal or other features co-occur with changes in aerodynamics.  If we move away from a false dichotomy of ``ground up'' versus ``trees down'', we are left with trying to understand flight itself, as the production of forces either for traction (WAIR), aerial maneuvering or weight support and propulsion; to all of these an deeper understanding of stability and control, enabled by what we have found here, is essential. 




















% Do NOT remove this, even if you are not including acknowledgments
\section{Acknowledgments}
We thank the following students, who also participated in the \Mgui project over the years: Chang Chun, Michael Cohen, Vincent Howard, Shyam Jaini, Felicia Linn, Divya Manohara,  Francis Wong, Karen Yang, Olivia Yu, and Richard Zhu. This research was done through the Berkeley Undergraduate Research Apprentice Program (URAP). We also thank Robert Dudley for occasional use of his wind tunnel, Tom Libby and the Berkeley Center for Integrative Biomechanics in Education and Research (CIBER) for use of a force sensor.  We are sad to have lost one member of our team to tragedy, Alex Lowenstein, and are grateful for our time with 
him.  

\index{Microraptor|)}

% copy all citations from this to main bibliography


