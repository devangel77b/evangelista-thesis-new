\rcsid{$Id$}
\rcsid{$Header$}
\rcskwsave{$Author$}
\rcskwsave{$Date$} 
\rcskwsave{$Revision$}

\chapter{\emph{Alectoris chukar} Animal Use Protocol}
\label{app:AUP}
\index{Chukar}
\index{Alectoris chukar@\textit{Alectoris chukar}}
\index{Animal Use Protocol|(}
The following are excerpts from UC Berkeley Animal Use Protocol R282, revision 1 for the use of Chukars (\emph{Alectoris chukar}) in studies of directed aerial descent and righting.

\section{Research goals}
\index{Animal Use Protocol!goals}
Existing paleobiological scenarios for the origin of flapping flight in bats, birds, and pterosaurs strongly implicate transition from a gliding and maneuvering form, but the biomechanical and aerodynamic correlates of this transition are unclear. To examine biomechanical constraints relevant to such a transition, recent and past work includes studies of falling geckos \citep{Jusufi:2008}, gliding frogs \citep{Emerson:1990, McCay:2001}, extinct feathered dinosaurs \citep{Xing:2003}, and flying squirrels.  Other parallel work in invertebrate taxa, such as gliding ants \citep{Yanoviak:2005} and gliding stick insects also strongly implicate transition from a gliding and maneuvering form.

An alternate scenario for the origin of flapping flight in vertebrates involves the use of wings to assist running and traction up steep terrain \citetext{wing-assisted incline running, \citealp{Dial:2003, Bundle:2003, Dial:2008}}.  Studies in support of this alternative scenario have observed that hatchling birds utilize flapping movements when running up inclines \citep{Dial:2008}.   However, these studies ignore use of the wings during other aerial-related behaviors, for example, use of the wings when descending vertically.  Using the same species and general experimental setup, we plan to address this gap by determining limb and body kinematics, both symmetric and asymmetric, that contribute to aerial righting and directed aerial descent maneuvers, and that may have historically led to bilateral limb flapping in birds.  Wind tunnel studies of static and flapping models and computer simulations based on aerodynamics and inertial mechanics will complement these kinematic studies.


\section{Justification for animal use}
\index{Animal Use Protocol!justification for animal use}
\subsection{Rationale for use of animals}
Flying and gliding animals are paradigmatic examples of the generation and control of unsteady aerodynamic forces, and exhibit both neuromuscular regulation and multimodal sensory integration that far surpass current technological capacities.  To understand both generation and control of these aerodynamic phenomena, it is necessary to study living animals as they naturally locomote in the air.

\subsection{Rationale for choice of species and numbers}
Chukar Partridges are a model system for wing-assisted inclined running; as this work examines a gap in WAIR theory and seeks to extend it, they are a logical choice to start with as the data obtained will be directly comparable to previous studies \citep{Bundle:2003, Dial:2003, Dial:2008}.  Chukars are widely available through the poultry trade \citep{Heinrichs:2009, OToole:2003, Willis:2009}.  The number of study birds is based on our lab�s previous work in similar kinematic studies and should provide sufficient replicate measurements. \index{Animal Use Protocol!species rationale}\index{Animal Use Protocol!number rationale}


\section{Description of laboratory research}
\index{Animal Use Protocol!research description}
With Chukar Partridges, we seek to determine 1) the presence or absence of an aerial righting reflex over ontogeny; 2) the presence or absence of directed aerial descent ability over ontogeny; 3) three-dimensional trajectories and limb and tail usage during such maneuvers; and 4) the impact of a limited set of non-invasive manipulations (attachment/augmentation of feathers, especially pelvic wing feathers \citep{Lippincott:1920, Xing:2003, Thomas:1997, Evans:1992, Evans:1994} and augmentation of tail inertia \citep{Jusufi:2008}.  

All experiments will involve filming with video cameras illuminated with \SI{500}{\watt} lights.  Each filming event lasts up to thirty seconds.  Birds may be filmed on a daily basis for periods of three hours  During all experiments, animals will be observed for signs of weakness and will be removed from the study if such signs are evident.  Animals showing signs of reluctance will be given time to habituate to experimental setups.  Animals may be non-destructively marked by either attachment of adhesive-backed \SI{3}{\milli\meter} reflectors  or use of Wite-out and black marker, both typical in other studies of bird locomotion \citep{Dial:2008, Hedrick:2007, Daley:2007, Essner:2002, Wischusen:1989}.  During marking, animals will be restrained by hand.


\subsection{Presence or absence of aerial righting reflexes over ontogeny}
Chicks will be placed on a platform such as a ladder and allowed to take off freely, or dropped at a random orientation by tipping out of a cup.  Their vertical orientation will be observed during descent using high-speed video recording of kinematics.  Gentle vibration may be applied to the cup to induce takeoff.  The experimental setup will provide for a soft landing area, such as a loosely spanned, soft and eleastic cloth or foam.  The methods here will be identical to those we have used to study aerial righting and directed aerial descent in rain forest canopy ants, stick insects, and also in geckos \citep{Jusufi:2008}.\index{Animal Use Protocol!kinematics}

All chukar experiments will be conducted using a ladder, ramp, or scaffold structure within a \SI{5 x 3}{\meter} full-ceiling-height animal enclosure with fabric or netting walls in Haas 97/99.  For runs in which voluntary bird behavior is recorded, filming will be conducted up to daily for up to three hours per day.  For runs in which birds are gently stimulated, \SI{5}{\minute} rest periods will be provided between glides and a \SI{30}{\minute} rest every five glides, with an absolute maximum of 15 glides per animal per day. \index{Animal Use Protocol! drop tests} 

\subsection{Presence or absence of directed aerial descent ability over ontogeny}
Using methods similar to \citep{Dial:2008}, chicks will be allowed to run up an incline and jump off it freely; or will be placed at the top of an obstacle and gently stimulated to descend from it into a soft landing area. Filming of the descent with multiple high-speed video cameras will assess if trajectories show evidence of turns or if they are random or confined to a single plane \citep{Essner:2002, Socha:2002}.  Most runs will film the free, volitional behavior of chicks as they explore the experimental setup.  

To provide additional testing of the extent of directed aerial descent abilities, the target landing zone may be displaced over small distances after the chick jumps, as has been done in what was done in previous studies of flying squirrels \citep{Wischusen:1990}.

\subsection{Three-dimensional trajectories and limb and tail usage during such maneuvers}
Part of this experiment will be conducted concurrently with experiments 1 and 2, which already film the animals using multiple high-speed video cameras that are sufficient to obtain three-dimensional trajectories and appendage use during maneuvers.  

To obtain additional information on aerodynamic use of appendages , chicks may be placed in the working section of a vertical wind tunnel  to simulate conditions of free fall \citep{McCay:2001, Jusufi:2008}. Equilibrium gliding at terminal falling velocity is reached when the aerodynamic drag and lift forces balance the force of gravity. A small animal like a Chukar Partridge will attain terminal velocity at a ventral airflow of less than \SI{6}{\meter\per\second}, depending on individual mass and surface area.  Chicks will not be exposed to air speeds exceeding the equivalent of individual terminal velocity. To prevent chicks from maneuvering sideways out of the test section and to enable high-speed video filming, transparent acrylic sidewalls will be mounted around the opening of the wind tunnel. A safety net will be installed in the test section to prevent animals from contacting the expansion chamber of the wind tunnel.\index{Animal Use Protocol! use of wind tunnel}


\subsection{Relative effects of inertia and aerodynamic forces in maneuvers}
 To examine the role of inertia, small weights no more than \SI{10}{\percent} of body weight will be attached to the chicks using veterinary wrap, similar to methods used in \citep{Daley:2007}.  Inertia will be increased by addition of a ``prosthetic tail'' made from a lightweight shaft (e.g. music wire, wood, plastic or cut turkey feathers) with a small weight held onto to the chick�s natural tail using veterinary wrap.  For control purposes, an equivalent amount of weight may be added at the hips, near the center of mass, or on a leg or wing \citetext{as in \citealp{Daley:2007}}. Weights will be removed at the end of each session. index{Animal Use Protocol!inertial augmentation}
 
To examine the role of aerodynamic forces, we will observe aerial behaviors over ontogeny as the bird�s natural feathers develop.  In addition, we may clip the primary feathers and retrices, augment the primary feathers or retrices by gluing of additional feather extensions \citetext{\citealp{Evans:1994}, approved UCB Animal Use Protocol R282}, or augment feathers by gluing flight feathers at the position of other, non-flight feathers such as the pelvic ``wing'' plumage \citep{Lippincott:1920}.
 
Feather extensions will be conducted using the method of \citet{Evans:1994}.  Feathers will be cut near the base and new feathers glued on to vary length from between \SIrange{5}{10}{\percent} of the original length.  Feather extensions will be glued using a combination of pins and cyanoacrylate superglue.  Attached feathers will have been frozen several months to kill any parasites that may have been present. 
During these manipulations, chicks will be restrained by hand.  No anesthetization is necessary because the manipulations involve no living tissue; no living tissue is manipulated other than whole-body restraint for no more than \SI{10}{\minute} during these procedures.  Feathers are attached carefully so that they retain aerodynamic function; this method has been used extensively for other avian taxa \citep{Evans:1994, Evans:1992, Thomas:1997} and these authors report that manipulated birds folded their tails naturally and did not pick at or seem to unduly notice the manipulated feathers.  Maneuverability and aerodynamic performance of individuals with manipulations will be assessed using the methods described above.  Upon completion of manipulation experiments, manipulated feathers will be plucked to induce their replacement.  Birds that pick at extensions will have the extensions checked and adjusted as practicable and will be given time to acclimate, but extensive picking or grooming of the extensions may invalidate the experiment and such birds will be removed from study. \index{Animal Use Protocol!feather extensions}

\section{Method of euthanasia and disposition of specimens}
\index{Animal Use Protocol!euthanasia}
Euthanasia, if needed for Chukar Partridges at study's completion: overdose of isoflurane or carbon dioxide inhalation followed by bilateral thoracotomy.

\section{Proposed animal housing}
Chukar Partridges will be maintained within the Animal Behavior Research Suites on the fifth floor of VLSB.  Chukar Partridge chicks will be maintained within the Animal Behavior Research Suites on the fifth floor of VLSB.  Birds will be kept on brooder bedding litter (wood shavings, sawdust, compressed wood pellets, or other suitable material) changed bi-weekly or as necessary \citep{Heinrichs:2009, Willis:2009}.  Lamps will be provided to maintain a warm temperature as necessary \citep{Heinrichs:2009, Willis:2009}.  Birds will be kept in an enclosure with approximately 25 chicks to a \num{4 x 3} foot area \citep{Heinrichs:2009, Willis:2009}.  We will keep an individual Chukar Partridge for up to eight weeks, to allow completion study up to the point of being fully feathered and slightly beyond.  Most work will be completed at approximately four weeks.  Batches of chicks will not be mixed and additional space will be provided as birds age beyond four weeks (approximately \num{2} square feet per bird) \citep{Heinrichs:2009, Willis:2009}. \index{Animal Use Protocol!housing}

Chukar Partridge will be fed typical chick starter rations (\SI{20}{\percent} protein) in suitably designed feeding containers \citep{Heinrichs:2009, Willis:2009}.  OLAC personnel will feed the birds daily following standard UCB arrangements.  Diet will occasionally be supplemented with grit, vegetable material, or insect larvae \citep{Heinrichs:2009, Willis:2009}.  Water will be available at all times in suitably designed watering containers no more than a few inches deep \citep{Heinrichs:2009, Willis:2009}. \index{Animal Use Protocol!feeding}

When Chukar Partridges are returned to fifth floor housing after flight experiments, they will be monitored one hour after return and again the following morning to ensure normal behavior.  We typically check in on all of our animals one to two times daily independent of the occurrence of flight experiments.
	
\section{Breeding}
No breeding will be undertaken. \index{Animal Use Protocol!breeding}

\section{Capture and transportation of animals}
Chukar Partridges will be obtained as one-day-old chicks and shipped via standard shipping methods for poultry \citep{Heinrichs:2009, Willis:2009}.   For experiments, chicks will be transported between the Animal Behavior Research Suites on the fifth floor of VLSB and Haas 97/99 (a one minute walk) using a small animal carrier with litter and provision for ventilation.  No more than 12 chicks will be placed in one box (or less depending on size).  \index{Animal Use Protocol!transport of birds}

\section{Description of field research}
No field components are associated with studies of Chukar Partridges.\index{Animal Use Protocol!field research}

\index{Animal Use Protocol|)}

